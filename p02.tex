\section{Кратчайшие пути}

% \subsection{Практика}
% \begin{enumerate}
%   \item
%     Дан орграф с положительными весами на ребрах. Найти кратчайший путь, проходящий по всем $k$ выделенным вершинам. 
%     Время $\O(2^k E \log{(2^k V)})$.

%   \item
%     Пусть даны $n$ валют и $m$ обменников, и $i$-й обменник предлагает менять
%     валюту $a_i$ на валюту $b_i$ по курсу $c_i/d_i$.  
%     Приведите алгоритм, определяющий за время $\O(nm)$, возможно ли бесконечно обогащаться.
%     Другими словами, существуют ли такие $p, q, s \: (p < q)$, что, имея $p$ единиц валюты типа $s$, 
%     можно за несколько обменных операций получить хотя бы $q$ единиц валюты типа $s$?
%     Считайте, что у обменников есть бесконечное количество денег
%     целевой валюты.

%   \item
%     Пусть дан взвешенный граф $G$ (возможно, с циклами отрицательного веса). На
%     вершинах этого графа определим функцию $\phi(v)$ --- потенциал.
%     Заменим вес каждого ребра $w(u, v)$ на $w'(u,v) = w(u, v) + \phi(u) - \phi(v)$.
%     Докажите, что кратчайшим путям между двумя вершинами в
%     графе с весами $w'$ будут однозначно соответствовать кратчайшие пути в графе с
%     весами $w$.

%   \item
%     Пусть во взвешенном графе $G$ нет циклов отрицательной
%     стоимости. Докажите, что если в качестве потенциала взять
%     кратчайшее расстояние от некоторой вершины $s$, то все веса $w'$
%     получатся неотрицательными (если соответствующие расстояния конечны).

%   \item
%     Дана система из $m$ неравенств на $n$ переменных $x_i$. Каждое
%     неравенство имеет вид $x_j - x_i \leq \delta_{ij}$.
%     \begin{enumerate}
%       \item Найти решение системы или сказать, что его не существует, за $\O(n \cdot m)$.
%       \item Пусть все $\delta_{ij} \ge 0$, решить задачу за $o(n \cdot m)$.
%     \end{enumerate}

%   \item
%     Дан произвольный взвешенный орграф, найдите расстояние от $a$ до $b$
%     (число или $\pm \infty$). $\O(V E)$.

%   \item
%     Предподсчёт за $\O(V^3)$ и запрос $\langle a, b, e \rangle$ за
%     $\O(1)$ --- существует ли кратчайший путь из $a$ в $b$, проходящий
%     через ребро $e$?

%   \item
%     Для каждой вершины графа узнать, есть ли отрицательный цикл через
%     эту вершину. $\O(V^3)$.

%   \item
%     Для каждой пары вершин в графе найти $w[a,b]$ -- такой минимальный
%     вес, что из $a$ в $b$ есть путь по рёбрам, вес которых не больше
%     $w[a,b]$. $\O(V^3)$.

%   \item
%     Нужно научиться на запрос ``уменьшился вес ребра'' за $\O(V^2)$
%     пересчитывать матрицу расстояний. Считайте, что в графе не было и не появилось отрицательных циклов.

%   \item
%     Рассмотрим граф, на каждом ребре которого написан нолик или единица.
%     Каждому пути в этом графе соответствует некоторая бинарная
%     строка. Строки сравниваются как двоичные числа (старшие разряды слева),
%     среди одинаковых чисел меньше то, у которого меньше ведущих нулей.
%     Для задачи нахождения кратчайших путей относительно такой метрики докажите
%     корректность или приведите контрпример для алгоритма
%     \begin{enumerate}
%       \item Флойда
%       \item Дейкстры
%       \item Беллмана-Форда
%     \end{enumerate}

% \end{enumerate}

% \newpage
\subsection{Домашнее задание}
\begin{enumerate}
  \item
    Дан взвешенный неориентированный граф $G = \langle V, E \rangle$ с положительными весами.
    В графе есть подмножество вершин $T \subset V$, которые мы назовем терминалами.
    Минимальное дерево Штейнера --- это связный подграф графа $G$ минимального веса, содержащий
    все терминалы.
    \begin{enumerate}
      \item
        Пусть граф полный, и все веса удовлетворяют неравенству треугольника $c_{ij} + c_{jk} \geq c_{ik}$.
        Найдите дерево Штейнера, которое не больше, чем в 2 раза превышает размер минимального,
        (т.е. 2-приближение) за $\O(V^2)$.
      \item
        Пусть граф связный, но неравенства треугольника нет. Найдите 2-приближение за $\O(V^3)$.
    \end{enumerate}

  \item
    Пусть на вершинах графа задан порядок: $v_1, v_2, \cdots, v_n$.
    Пусть алгоритм Беллмана-Форда на каждой стадии рассматривает
    ребра в таком порядке: сначала ребра, ведущие из меньшей вершины в
    большую (в порядке возрастания исходящей вершины), а потом ребра,
    ведущие из большей вершины в меньшую (в порядке убывания исходящей
    вершины). Докажите, что если в графе нет циклов отрицательного
    веса, то алгоритм найдет все кратчайшие пути из $v_1$ за $\frac{n}{2}$
    стадий.

  \item
    Дан взвешенный орграф $(V, E)$, посчитайте от начальной вершины $s$ до всех остальных
    кратчайшие расстояния относительно метрики:
    \begin{enumerate}
      \item <<вес самого \texttt{тяжелого} ребра на пути>>. $\O(E \log{V})$. \\
        \textit{(Подсказка: модифицируйте алгоритм Дейкстры и докажите, что он работает)}
      \item <<вес самого \texttt{легкого} ребра на пути>>. $\O(V + E)$.
    \end{enumerate}

% \subsection*{Дополнительные задачи}

%   \item
%     Докажите для задачи 2 оценку $\frac{n}{3}$ в среднем, если вершины пронумерованы случайной перестановкой.

\end{enumerate}

