\section{Жадные алгоритмы}

% \subsection{Практика}
% \begin{enumerate}
%   \item
%     Постройте код Хаффмана за $\O(n)$, если частоты букв уже даны в отсортированном порядке.

%   \item
%     В фирму поступило $n$ заказов, которые можно выполнять в произвольном порядке. На выполнение
%     заказа $i$ необходимо время $t_i$. В каждый момент времени можно работать ровно над одним заказом.
%     Пусть $e_i$~-- момент окончания выполнения заказа номер $i$. Распределите работу над заказами
%     так, чтобы минимизировать $\sum_i e_i$. Время $\O(n \log n)$.

%   \item
%     Есть $n$ работ, у каждой есть $t_i > 0$ --- время ее выполнения и $d_i$ "--- дедлайн.
%     \begin{enumerate}
%       \item Определите, можно ли выполнить их все. $\O(n \log n)$.
%       \item Найдите максимальное подмножество, которое можно выполнить. $\O(n \log n)$.
%     \end{enumerate}

%   \item
%     Даны $n$ гномов. Если $i$-го гнома укладывать спать $a_i$ минут, он потом спит $b_i$ минут.
%     Можно ли сделать так, чтобы в какой-то момент все гномы спали? $\O(n \log{n})$.

%   \item
%     Даны $n$ монеток, у каждой есть своя вероятность выпадения орла
%     $p_i$. Нужно выбрать подмножество размера $k$ из них, для которого
%     вероятность выпадения ровно $\frac{k}{2}$ орлов при одновременном
%     подбрасывании максимальна ($k$ --- четное).
%     \begin{enumerate}
%       \item $\O(n \log{n} + k^3)$.
%       \item $\O(n + k^2)$.
%     \end{enumerate}

%   \item
%     Машина тратит единицу топлива на километр, имеет бак
%     объема $k$ и находится в начале прямой дороги в точке $0$. Для всех $i \in \mathbb{N} \cup \{0\}$ 
%     в $i$ километрах от неё есть заправочная станция со своей
%     положительной ценой $c_i$. Определите за время $\O(n)$, как 
%     проехать $n$ километров за минимальную стоимость. 

%   \item
%     Маршрутка совершает рейс от первой до $n$-й остановки. В маршрутке
%     $m$ мест для пассажиров.  Есть $k$ человек, про каждого заранее
%     известно, что он хочет доехать от остановки $s_i$ до $f_i$. Проезд
%     для пассажира стоит $1$ вне зависимости от расстояния между
%     остановками. Максимизируйте прибыль, при условии, что можно
%     выбирать, кого сажать в маршрутку на каждой остановке. $\O((n + m + k) \log{m})$.

% \subsection*{Дополнительные задачи}

%   \item
%     Будем называть \textit{независимым множеством} (или \textit{антикликой}) подмножество вершин
%     графа, в котором нет ребер. Пусть в графе $G$ есть $V$ вершин и $E$ ребер, а максимальная
%     степень равна $d$. Найдите в нём независимое множество размера хотя бы:
% 	\begin{enumerate}
% 	  \item $\frac{n}{d+1}$ за время $\O(V + E)$
% 	  \item $\sum\limits_{v\in V(G)} \frac{1}{\deg(v)+1}$ за время $\O(V \log{V} + E)$
% 	\end{enumerate}
% 	Считайте, что граф уже дан в памяти в виде массива, где для каждой вершины хранится список её соседей.

%   \item
%     Имеется $n$ деталей и два станка. Каждая деталь должна сначала пройти обработку на первом станке,
%     затем --- на втором. При этом $i$-ая деталь обрабатывается на первом станке за $a_i$ времени,
%     а на втором — за $b_i$ времени. Каждый станок в каждый момент времени может работать только с одной деталью.
%     Требуется составить такой порядок подачи деталей на станки, чтобы итоговое время обработки всех деталей было
%     бы минимальным. $\O(n \log n)$.

% \end{enumerate}

% \newpage
\subsection{Домашнее задание}
\begin{enumerate}
  \item
    \begin{enumerate}
      \item
        Пусть в тексте встречается символ с частотой большей $\frac{2}{5}$.
        Доказать, что любой оптимальный (бинарный) префиксный код будет содержать слово длины 1.

      \item
        Пусть в тексте все символы встречаются реже $\frac{1}{3}$.
        Доказать, что при кодировании любым оптимальным префиксным кодом каждое кодовое слово 
        будет иметь длину не меньше 2.
    \end{enumerate}

  \item
 	Преподаватели сделали $n$ заявок на занятие. Каждое
 	занятие начинается в момент $b_i$ и кончается в момент $e_i$
 	(занимает интервал $[b_i, e_i)$). Два занятия в одной аудитории
 	быть не могут. Распределите заявки по аудиториям так, чтобы
 	общее число аудиторий было минимально. Решить за $\O(n \log n)$.

  \item
    $n$ aтлетов хотят выстроить из своих тел башню максимальной
    высоты. Башня это цепочка атлетов, первый стоит на земле, второй
    стоит у него на плечах, третий стоит на плечах у второго и
    т.д. Каждый атлет характеризуется силой $s_i$ и массой $m_i$. Сила
    это максимальная масса, которую атлет способен держать у себя на
    плечах. Известно, что если атлет тяжелее, то он и сильнее, но
    атлеты равной массы могут иметь различную силу. Определите
    максимальную высоту башни. $\O(n \log n)$.

    \begin{solution}
        Будем строить башню тел сверху вниз. Отсортируем массив пар (масса $m_i$, сила $s_i$) за $\O(n \log n)$ по возрастанию. Будем поддерживать переменную $cur\_weight$ -- текущую массу всей башни, а также массив-ответ, в который будем добавлять индексы атлетов, которые будут формировать ответ для задачи. Жадный шаг: возьмем самого легкого атлета и добавим его массу к $cur\_weight$. Дальше в отсортированном массиве сил $s_i$ найдем первого (бинпоиск), чья сила $s_i$ будет больше либо равна текущей массы башни $cur\_weight$ (чтобы он точно смог взять на свои плечи всех до него) и добавим его в массив-ответ, а еще прибавим его массу к $cur\_weight$. Дальше будем действовать точно также: ищем первого подходящего (самого легкого) под условие $s_i \geq cur\_weight$, добавляем его в массив-ответ и прибавляем его массу к $cur\_weight$. В конечном итоге максимальная высота башни будет равна длине массива-ответа. 
        
        Корректность: сравним наше решение с оптимальным: имеем наш массив индексов и оптимальный массив индексов, смотрим на элементы, пока они не совпадут; если совпадают, то наше решение совпадает с оптимальным, а если появляется различие, то смотрим на массы -- так как наш жадный шаг заключался в том, чтобы взять самого легкого, который может поднять всю башню перед ним, то наше решение будет не хуже оптимального.

        Асимптотика: сортируем за $\O(n \log n)$, потом проходимся по массиву ($\O(n)$) и ищем следующих ($\O(\log n)$), в конце возвращаем длину массива-ответа ($\O(1)$). Суммарно $\O(n \log n)$.
    \end{solution}

  \item
	Есть $n$ человек. Человек $i$ готов примкнуть к нашей банде, если наш авторитет
	хотя бы $a_i$, при этом он изменит наш авторитет на $b_i$. Наш изначальный авторитет
	равен $A$. $a_i, b_i, A \in \mathbb{Z}$
	\begin{enumerate}
		\item Можем ли завербовать всех людей? $\O(n \log n)$.
		
        \begin{solution}
            Для начала разделим массив $b_i$ на два $b1_i, b2_i$. В $b1_i$ будут лежать все $b_i$, для которых $b_i \geq 0$, а в $b2_i$ -- $b_i$, для которых $b_i < 0$. Дальше отсортируем $b1_i$ по возрастанию $a_i$ и будем брать их всех по порядку. Это выгодно, потому что на каждом шаге мы забираем в банду человека с минимальными требованиями по авторитету, который в свою очередь прибавляет свой $b_i > 0$ к нашему авторитету $A$. Если в какой-то момент мы не сможем взять очередного человека, то это и будет максимальное количество людей, которых можно взять в банду. Больше не получится, потому что авторитет невозможно уже никак поднять, потому что человек, который может его поднять и в то же время с минимальными требованиями по авторитету недоступен.

            Если мы смогли взять всех тех людей, которые вносят положительный авторитет, то, после примыкания всех людей, вносящих положительный $b_i$ у нас образовался некоторый $A' \geq A$. Теперь будем работать со вторым массивом $b2_i$, в котором каждый человек будет уменьшать наш текущий авторитет, потому что $b_i$ в $b2_i$ отрицательные. Отсортируем его по возрастанию модуля $|b_i|$ и по уменьшению $a_i$ (если есть два человека, понижающих авторитет на одну и ту же величину $|b_i|$, то выгодно сначала взять того, у кого требование по авторитету выше, потому что в противном случае, взяв человека с меньшими требованиями и понизив из-за него авторитет, мы потом можем не смочь взять человека с требованиями выше!). Будем пытаться брать всех по порядку, если их требования по авторитету меньше текущего авторитета. Так выгодно брать потому, что уменьшение $A'$ на каждом шаге минимально, поэтому мы сможем набрать как можно большее число людей. Если в какой-то момент мы дошли до конца массива и больше не можем никого взять (на предыдущих шагах не брали из-за несоответствия требованиям по репутации), то это и есть наш максимум. Больше взять не получится, потому что в таком случае мы бы еще сильнее занижали нашу репутацию, беря в банду людей, которые больше занижают авторитет и тем самым потенциально сокращают число людей, которых возможно взять.

            Корректность. Рассмотрим этап с массивом положительных $b_i$: наш алгоритм не хуже оптимального, потому что он на каждом шаге берет наиболее доступного (с самыми низкими требованиями по авторитету) человека, который вносит положительный вклад в репутацию. Рассмотрим этап с массивом отрицательных $b_i$: наш алгоритм не хуже оптимального, потому что он на каждом шаге берет человека, который меньше всего может понизить нашу репутацию, и, как следствие, меньше всего сузить количество доступных для вербовки людей.

            Асимптотика: разбиваем на разные массивы за $\O(n)$, сортируем массивы за $\O(n \log n)$, проходим по массивам и набираем банду за $\O(n)$, возвращаем ответ за $\O(1)$. Суммарно $\O(n \log n)$.
        \end{solution}

		\item Какое максимальное число людей мы можем завербовать? $\O(n \log n)$.
		
        \begin{solution}
            Для начала разделим массив $b_i$ на два $b1_i, b2_i$. В $b1_i$ будут лежать все $b_i$, для которых $b_i \geq 0$, а в $b2_i$ -- $b_i$, для которых $b_i < 0$. Дальше отсортируем $b1_i$ по возрастанию $a_i$ и будем брать их всех по порядку. Это выгодно, потому что на каждом шаге мы забираем в банду человека с минимальными требованиями по авторитету, который в свою очередь прибавляет свой $b_i > 0$ к нашему авторитету $A$.

            После примыкания всех людей, вносящих положительный $b_i$ у нас образовался некоторый $A' \geq A$. Теперь будем работать со вторым массивом $b2_i$, в котором каждый человек будет уменьшать наш текущий авторитет, потому что $b_i$ в $b2_i$ отрицательные.



            Асимптотика: разбиваем на разные массивы за $\O(n)$, сортируем массивы за $\O(n \log n)$. Суммарно $\O(n \log n)$.
        \end{solution}
	\end{enumerate}


% \subsection*{Дополнительные задачи}

%   \item
%     Будем называть \textit{независимым множеством} (или \textit{антикликой}) подмножество вершин
%     графа, в котором нет ребер. Пусть в графе $G$ есть $V$ вершин и $E$ ребер, а максимальная
%     степень равна $d$. Найдите в нём независимое множество размера хотя бы:
% 	\begin{enumerate}
% 	  \item $\frac{n}{d+1}$ за время $\O(V + E)$
% 	  \item $\sum\limits_{v\in V(G)} \frac{1}{\deg(v)+1}$ за время $\O(V \log{V} + E)$
% 	\end{enumerate}
% 	Считайте, что граф уже дан в памяти в виде массива, где для каждой вершины хранится список её соседей.

%   \item {\bf Пятачок, у тебя есть дома ружье?}\\
% 	Даны $n$ непересекающихся кругов на плоскости. Мы стоим в точке $(0,0)$ и можем стрелять по прямой.
% 	Минимальным числом выстрелов проткнуть все круги.

%   \item
% 	$n$ школьников упали в яму глубины $S$. Каждый школьник имеет рост (от ног до плеч) $h_i$
% 	и длину рук $l_i$. Школьники могут вставать друг другу на плечи, верхний школьник может
% 	вытянуть руки. Чтобы выбраться из ямы необходимо дотянуться руками до уровня земли.
% 	\begin{enumerate}
% 		\item Могут ли выбраться все школьники? $\O(n \log n)$.
% 		\item Какое максимальное число школьников может выбраться? $\O(n \log n)$.
% 	\end{enumerate}

%   \item
%     В фирму поступают заказы, которые можно выполнять в произвольном
%     порядке. В каждый момент времени можно работать ровно над одним
%     заказом.  Изначально заказов нет, $i$-й заказ поступает в момент
%     времени $r_i$, работать над ним нужно $t_i$. Пусть $e_i$~-- момент
%     окончания выполнения заказа номер $i$. Распределите работу над
%     заказами так, чтобы минимизировать $\sum_i e_i$. Переходить от
%     одного заказа к другому можно в любой момент времени (даже если
%     заказ не доделан до конца, незаконченный заказ можно будет 
%     возобновить с того же места). 
%     Свойства заказа ($r_i, t_i$) не известны до
%     момента его поступления. Время $\O(n \log n)$, при условии, что
%     всего поступит $n$ заказов.

%   \item
%     \textit{Раскраской вершин} графа $G = (V, E)$ называется функция $c: V \to [m]$, сопоставляющая каждой вершине $G$ цвет от $1$ до $m$. 
%  	Раскраска называется \textit{правильной}, если каждая пара соседних вершин имеет разные цвета. 

%  	Для неориентированного графа $G = (V, E)$ его \textit{хроматическим числом} $\chi(G)$ называется наименьшее возможное число цветов в правильной раскраске $G$.

%  	Для графа $G$ обозначим размер его максимального полного подграфа через $\omega(G)$.

%  	Рассмотрим на вещественной прямой замкнутые отрезки $I_1, I_2, \dots, I_n$. Сопоставим каждому отрезку $I_i$ вершину $v_i$ 
%  	и каждой паре пересекающихся отрезков $(I_i, I_j)$ ребро $(v_i, v_j)$. Такой граф будем называть \textit{интервальным графом}.

%  	Будем называть граф $G$ \textit{совершенным}, если для любого его индуцированного подграфа $H$ верно $\omega(H) = \chi(H)$. 

%  	Докажите, что каждый интервальный граф совершенен. Приведите алгоритм, красящий интервальный граф $G = (V, E)$ с $|V| = n$ в $\omega(H)$ цветов за время $O(n \log(n))$.

%   \item
%  	В фирму поступают заказы, которые можно выполнять в произвольном
%  	порядке. В каждый момент времени можно работать ровно над одним
%  	заказом.  Изначально заказов нет, $i$-й заказ поступает в момент
%  	времени $r_i$, работать над ним нужно $t_i$. 

%  	Все заказы объединены в проекты (один заказ относится к одному
%  	проекту, заказы из одного проекта могут поступать не подряд).

%  	Пусть $e_i$~-- момент окончания выполнения последнего (в порядке
%  	выполнения) из заказов в проекте с номером $i$.

%  	Нужно распределить работу над заказами так, чтобы минимизировать
%  	$\sum_i e_i$. Свойства заказа ($r_i, t_i$, его проект) не известны
%  	до момента его поступления.  Переходить от одного заказа к другому
%  	можно в любой момент времени (даже если заказ не доделан до конца).

%  	Придумайте решение, которое не более чем в два раза хуже
%  	оптимального.  Время $\O(n \log n)$, при условии, что всего поступит $n$ заказов.

\end{enumerate}
