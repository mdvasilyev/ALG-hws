\section{Деревья}

% \subsection{Практика}
% \begin{enumerate}

%   \item
%     Запросы online за $\O(\log n)$:
%     \begin{itemize}
%       \item \texttt{add(i, x)} --- вставить $x$ на позицию $i$, все элементы после него сдвигаются на $1$ вправо
%       \item \texttt{del(i)} --- удалить элемент на позиции $i$, все элементы после него сдвигаются на $1$ влево
%       \item \texttt{sum(l, r)} --- сумма всех $x$, для которых $l \le i \le r$
%       \item \texttt{add(l, r, value)} --- добавить $value$ ко всем $x$, для которых $l \le i \le r$
%     \end{itemize}

%   \item
%     Нарисуйте все Декартовы деревья, которые могут получится в результате
%     операции \texttt{merge(бамбук идущий влево-вниз, вершина)} и
%     \texttt{merge(бамбук идущий вправо-вниз, вершина)}.

%   \item
%     Пусть приоритеты случайны. Покажите, что наличие одинаковых ключей не портит оценку на
%     матожидание высоты Декартова дерева для последовательности стандартных операций \texttt{insert}.

%   \item
%     \begin{enumerate}
%       \item
%         Попробуем улучшить константу у стандартного \texttt{insert} следующим образом: сначала
%         будем спускаться по дереву, пока не встретим узел с меньшим, чем у нового, приоритетом,
%         и уже этому поддереву сделаем \texttt{split} и подвесим две части к новому узлу.
%         Покажите, что результате всегда получится корректное Декартово дерево.

%       \item
%         Придумайте аналогичный \texttt{delete} через спуск и один \texttt{merge}.

%       \item
%         Пусть теперь приоритеты случайны, а ключи могут быть одинаковыми. Заметим, что при спуске
%         у нас есть выбор, в какое поддерево идти при равенстве ключей. Влияет ли этот выбор на оценку
%         матожидания высоты дерева?
%     \end{enumerate}

%   \item
% 	Дана таблица $n \times n$ с целыми числами. Поступают запросы:
% 	\begin{itemize}
%       \item Прибавить значение к ячейке $(i, j)$,
% 	   \item Посчитать сумму чисел в прямоугольнике $((l_x, l_y), (r_x, r_y))$.
% 	\end{itemize}
%     $\langle \O(n^2), \O(\log^2 n) \rangle$, online.

%   \item
%     Дан массив из $n$ элементов, можно сделать предобработку за $\O(n \log n)$. Запросы: количество
%     различных чисел на отрезке $[L, R]$. \emph{Тут будет подсказка про $prev[i]$.}
%     \begin{enumerate}
%       \item online за $\O(\log^2 n)$.
%       \item offline за $\O(\log n)$.
%       \item online за $\O(\log n)$.
%     \end{enumerate}

%   \item
% 	Вывести все числа $\ge Y$ на отрезке $[L, R]$ массива.
% 	Static online
% 	\begin{enumerate}
%       \item $\langle \O(n \log{n}), \O(\log^2 n + k) \rangle$ ($k$ --- размер ответа).
%       \item $\langle \O(n \log{n}), \O(\log n + k) \rangle$.
%       \item Определите $k$ за $\langle \O(n \log{n}), \O(\log n) \rangle$
%         (разными способами).
% 	\end{enumerate}

%   \item
%     Запросы: $k$-е по порядку среди различных чисел на отрезке $[L, R]$. 
%     \begin{enumerate}
%       \item offline за $\O(\log^3 n)$.
%       \item online за $\O(\log^3 n)$.
%       \item online за $\O(\log^2 n)$.
%     \end{enumerate}

% \end{enumerate}

% \pagebreak
\subsection{Домашнее задание}
\begin{enumerate}

  \item
    Пусть приоритеты случайны, а ключи все разные. Найдите матожидание количества листьев в
    Декартовом дереве из $n$ вершин (точно, а не асимптотически).

  \item
    Пусть приоритеты случайны, а ключи все разные. Обозначим за $x_k$ вершину Декартова дерева,
    содержащую $k$-тый по порядку ключ. Всего в дереве $n$ вершин.
    \begin{enumerate}
      \item Пусть $1 \le i \le j \le k \le n$. Найдите вероятность того, что $x_j$ --- общий предок $x_i$ и $x_k$.
      \item Пусть $1 \le i \le k \le n$. Найдите матожидание длины пути между $x_i$ и $x_k$.
    \end{enumerate}
    Ответ достаточно выразить через комбинацию гармонических чисел $H_n = \sum\limits_{d = 1}^{n}\frac{1}{d}$.

  \item
    Даны число $K$ и изначально пустая последовательность. Вам поступает $n$ запросов, каждый одного из двух типов:
    \begin{itemize}
      \item \texttt{append(x)} --- дописать элемент $x$ в конец последовательности
      \item \texttt{rev()} ---
        развернуть $K$ последних элементов последовательности (если в данный момент их всего меньше $K$, то
        развернуть всю последовательность).
    \end{itemize}
    Вам нужно один раз после всех запросов вывести получившуюся последовательность.
    \begin{enumerate}
      \item $\O(n \log{n})$
      \item $\O(n)$
    \end{enumerate}

  \item \onlygroup{Мишунина}\\
    Дан массив из $n$ чисел. Все операции по модулю фиксированного $M$. Online запросы:
	\begin{itemize}
		\item посчитать произведение всех чисел на отрезке $[L, R]$;
		\item присвоить значение $x$ всем числам на отрезке $[L, R]$.
	\end{itemize}
	Убедитесь, что ваше решение работает именно за $\O(\log n)$.

  \item \onlygroup{Мишунина}\\
    Дана последовательность целых чисел $a_0, a_1, \dots, a_{n-1}$. Запросы:
    \begin{itemize}
      \item \texttt{assign(L, R, x)} --- присвоить $a_i \leftarrow x$ для $L \le i \le R$.
      \item
        \texttt{get(L, R)} --- найти $\max_{L \le l \le r \le R} \sum_{i=l}^r a_i$
        (т.е. максимальную сумму подотрезка $[L, R]$).
    \end{itemize}
    Online, $\O(\log{n})$ на запрос.

  \item \onlygroup{Кравченко и Крыштаповича}\\
    Дана пустая целочисленная плоскость. Нужно online за $\O(\log n)$ выполнять запросы:
    \begin{itemize}
      \item Добавить/удалить точку $(x_i, y_i)$
      \item Вывести любую точку из области $l_i < x < r_i,~ y < t_i$
    \end{itemize}

  \item \onlygroup{Кравченко и Крыштаповича}\\
    Дан массив из $n$ чисел. Нужно за $\O(\log n)$ обрабатывать запросы: количество
    различных чисел на отрезке $[L, R]$.
        \begin{itemize}
            \item offline, $\O(n)$ на предобработку;
            \item online, $\O(n \log n)$ на предобработку.
        \end{itemize}
    \emph{Подсказка: постройте массив $prev[i] = \max \Big\lbrace \{ j ~|~ j < i ~\land~ a[j] = a[i] \} \cup \{-1\} \Big\rbrace$}


% \subsection*{Дополнительные задачи}

%   \item \texttt{C-c-c-combo!}\\
%     Напишите на \texttt{Haskell} реализацию персистентной структуры,
%     умеющей отвечать на запросы:
%     \begin{itemize}
%     	\item \texttt{insert(i, x)}, $0 \le i \le L$ ($L$ --- текущая длина) --- вставить $x$ в позицию $i$. Все элементы в $i$ и правее сдвигаются на $1$ вправо.
%     	\item \texttt{delete(i)} --- удалить элемент из позиции $i$. Все правее сдвигается на $1$ влево.
%     	\item \texttt{add(l, r, value)} --- добавить $value$ ко всем $x$, для которых $l \le i \le r$
%     	\item \texttt{set(l, r, value)} --- установить в $value$ все $x$, для которых $l \le i \le r$
%     	\item \texttt{sum(l, r)} --- сумма всех $x$, для которых $l \le i \le r$
%     	\item \texttt{reverse(l, r)} --- изменить порядок всех $x$, для которых $l \le i \le r$, на обратный
%     \end{itemize}
%     В начальный момент структура пустая. Время работы на запрос $\O(\log{L})$, online.

\end{enumerate}
