\section{Сортировки и порядковые статистики}
% \subsection{Практика}
% \begin{enumerate}
%   \item
%     Дана последовательность из $n$ чисел, нужно за один проход и $\O(n)$ времени найти в ней
%     $k$ минимумов, используя $\O(k)$ дополнительной памяти.

%   \item
%     Работает ли аналогичная оценка времени работы детерминированного алгоритма
%     поиска порядковой статистики, если вместо пятерок разбивать элементы на
%     \begin{enumerate}
%       \item семерки.
%       \item тройки.
%     \end{enumerate}

%   \item
%     Придумайте, как добиться от \texttt{QuickSort} времени $\O(n \log{n})$
%     в худшем случае.

%   \item
%     Дан массив из n чисел от 1 до k, разработайте структуру данных, которая за $\O(1)$
%     отвечает на online запросы вида «Сколько в массиве элементов от a до b?». Время на
%     предподсчет $\O(n + k)$.

%   \item
%     Как с помощью цифровой сортировки сортировать строки над константным алфавитом
%     в лексикографическом порядке за $\O(\sum |s_i|)$?

%   \item
%     Даны $n$ точек на плоскости ($x_i, y_i$). Найти точку ($x^{*}, y^{*}$):
%     \begin{enumerate}
%       \item $\max\limits_i \Bigl[ \max(|x_i-x^{*}|, |y_i-y^{*}|) \Bigr] \rightarrow \min$
%       \item $\max\limits_i \Bigl[ |x_i-x^{*}|+|y_i-y^{*}|  \Bigr] \rightarrow \min$
%       \item $\sum\limits_i \Bigl[ (x_i-x^{*})^2+(y_i-y^{*})^2  \Bigr] \rightarrow \min$
%       \item $\sum\limits_i \Bigl[ |x_i-x^{*}|+|y_i-y^{*}| \Bigr] \rightarrow \min$
%       \item $\sum\limits_i \Bigl[ \max(|x_i-x^{*}|, |y_i-y^{*}|) \Bigr] \rightarrow \min$
%     \end{enumerate}

%   \item
% 	Даны два массива из положительных чисел $a$ и $b$, $|a| = |b| = n$.
% 	Выбрать массив $p$ из $k$ различных чисел от $1$ до $n$ так, чтобы
% 	$\frac{\sum_{i=1}^{k}a_{p_i}}{\sum_{i=1}^{k}b_{p_i}} \to \max$. %$\O(n \log{n})$.

% \subsection*{Дополнительные задачи}

%   \item
%     Возьмем массив a из n элементов, каждый из которых — это число $1$ до $n$.
%     Циклический сдвиг номер $i$ -- это последовательность
%     $a_i, a_{i+1}, \dots, a_{n-1}, a_0, a_1, \dots, a_{i-1}$. 
%     \begin{enumerate}
%       \item Предложите алгоритм сортировки циклических сдвигов в лексикографическом
%       порядке за время $O(n^2)$.
%       \item Предложите алгоритм сортировки циклических сдвигов за $\O(n \log n)$.
%     \end{enumerate}

%   \item
%     Пусть мы хотим показать, что некоторая сортирующая сеть сливает два отсортированных
%     массива в один. Покажите, что для этого достаточно показать, что она сливает массивы,
%     состоящие из $0$ и $1$.

% \end{enumerate}

% \pagebreak
\subsection{Домашнее задание}
\begin{enumerate}
  \item
    Есть улица длины $l$, которая освещается $n > 0$ фонарями, $i$-й фонарь находится в
    точке с вещественной координатой $a_i$. Фонарь освещает все точки улицы, которые находятся от него на расстоянии не больше d, где $d$ — некоторое
    положительное число, общее для всех фонарей.
    Найдите минимальное $d$, при котором вся улица освещена. $\O(n)$, $a_i$ не отсортированы.

    \begin{solution}
      На входе имеем массив неотсортированных координат $a_i$. Если каждую координату $a_i$ поделить на $l$, то становится удобно применить блочную сортировку (bucket sort). На лекции доказывали, что
      \begin{equation}
        \mathbb{E}\, T(n) = \O(n).
      \end{equation}
      За линейное время получили отсортированный массив $a_i$. Теперь соображения, которые помогут решить задачу: когда мы будем выбирать конкретное $d$, которое мы хотим минимизировать, мы должны учесть, что оно будет не меньше расстояния от начала улицы до первого фонаря, и оно будет не меньше расстояния от последнего фонаря до конца улицы, и оно будет не меньше половины максимального расстояния между ``внутренними'' фонарями. Половина здесь берется из-за того, что чтобы покрыть территорию между соседними фонарями, достаточно, чтобы каждый из этих фонарей мог покрыть только половину расстояния, тогда в сумме они покроют все расстояние. Так как мы получили отсортированные отнормированные $a_i$, то обозначим их как $a_i'$, а отнормированное $d$ как $d'$, тогда:
      \begin{equation} \label{distances}
        \begin{cases}
          d' \geq a_0' - 0, \\
          d' \geq 1 - a_n', \\
          d' \geq \max(\delta_i) / 2, \quad if n > 1 \Rightarrow \forall i \in [0, n - 1]
        \end{cases}
      \end{equation}
      где $\delta_i$ представляет собой расстояния между соседними фонарями. $\delta_i$ вычисляется через разность между $i+1$-м фонарем и $i$-м фонарем. При этом $i$ пробегает от 0 до $n - 1$. Решим систему неравенств (\ref{distances}) и получим минимальное $d_0'$. Теперь умножим $d_0'$ на $l$, чтобы вернуться к исходным размерам:
      \begin{equation}
        d_0 = d_0' \cdot l.
      \end{equation}
      Это расстояние $d_0$ и будет искомым расстоянием.

      \textit{Корректность}: пусть известен ответ $d_{ans}$. Покажем, что $d_0$, который выдает наш алгоритм, не хуже. Если фонарь только один, то $d_{ans}$ будет равен максимальному из расстояний от начала улицы до фонаря и от фонаря до конца улицы. Наш алгоритм получает точно такой же результат: если $n = 1$, то массив из одного элемента уже отсортирован, следовательно, в системе (\ref{distances}) остаются два первых неравенства. Решая систему неравенств будет выбрано $d_0'$, которое в данном случае будет представлять собой максимальное из расстояний от начала улицы до единственного фонаря и от этого фонаря до конца улицы. $d_0'$ умножается в конце на $l$. Понятно, что так как выбрано максимальное расстояние, оно покроет всю улицу целиком. Теперь, если фонарей $n \neq 1$: $d_{ans}$ будет больше или равен расстояния начала улицы до первого фонаря, а также он будет больше или равен от конца улицы до последнего фонаря, и еще он будет таким, чтобы покрыть расстояние между максимально удаленными соседними фонарями. Наш алгоритм получает точно такой же ответ: сначала алгоритм сортирует $a_i'$, потом подсчитывает $\delta_i$ -- расстояния между всеми соседними фонарями и отбирает максимальное из этих расстояний. После этого алгоритм выбирает минимальное подходящее $d_0'$, которое не меньше, чем расстояние от начала улицы до первого фонаря, и не меньше, чем расстояние от конца улицы до последнего фонаря, и не меньше, чем половина расстояния между максимально удаленными соседними фонарями. $d_0'$ умножается в конце на $l$.

      \textit{Асимптотика}: алгоритм сначала нормирует элементы массива за $\O(n)$, потом сортирует массив с помощью блочной сортировки за $\O(n)$, потом строит $\delta_i$ за $\O(n)$, потом выбирает максимум из $\delta_i$ за $\O(n)$, потом выбирает $d_0'$ за $\O(1)$ и в конце получает $d_0$ за $\O(1)$. В сумме:
      \begin{equation}
        T(n) = \O(n) + \O(n) + \O(n) + \O(n) + \O(1) + \O(1) = \O(n).
      \end{equation}
    \end{solution}

  \item
    Есть массив $a_i$, состоящий из $n$ неотрицательных целых чисел. Найти минимальное натуральное
    число, которого нет в массиве, за $\O(n)$ времени и $\O(1)$ дополнительной памяти.
    Входной массив доступен для записи. \\
    Замечание: $0 \leq a_i < 2^w$, где $w$ -- размер одной ячейки массива в битах.

    \begin{solution}
      Так как $a_i$ -- целые неотрицательные числа, то можно попытаться поработать с ними как с индексами. Будем при проходе расставлять элементы на свои места. Если попалась единица, то она должна стоять на нулевой позиции. Если, например, попалась пятерка, то она должна стоять на четвертой позиции из-за того, что индексы начинаются с нуля. Обозначим $a_i$ как $a[i]$. Тогда хочется, чтобы элемент $a[i]$ стоял на позиции $a[a[i]-1]$. При этом значение элемента (со сдвигом на 1) может быть больше размера массива, поэтому будем расставлять только те элементы, которые могут занять правильную позицию ($a[i] = a[a[i]-1]$). После прохода, на котором были расставлены (насколько возможно) $a[i]$ в правильные позицию, можно совершить еще один проход: начинаем с начала массива, и пока не обнаружим элемент, который не соответствует своей правильной позиции, идем до конца. Если на $i$-ой итерации стоит неправильный элемент ($a[i] \neq i + 1$), то возвращаем $i + 1$ в качестве ответа. Если дошли до конца массива, то возвращаем $n + 1$. Псевдокод, чтобы было понятнее:
      \begin{itemize}
        \item Цикл по $i$ от 0 до размера массива:
        
        \hspace{10mm} Пока $a[i]$ не превосходит размера массива и $a[i]$ не равно $a[a[i]-1]$ повторяем:

        \hspace{20mm} Свап $a[i]$ и $a[a[i]-1]$

        Цикл по $i$ от 0 до размера массива:

        \hspace{10mm} Если $a[i]$ не равно $i + 1$, то возвращаем $i + 1$

        Возвращаем $n + 1$
      \end{itemize}

      Корректность: допустим нам известен ответ, который можно получить следующим образом: отсортируем исходный массив и начнем проходить по нему с начала. Каждый следующий элемент будет превосходить текущий либо на 0, либо на 1, либо больше, чем на 1. Если превосходит на 0 или 1, то идем дальше, если больше, чем на один, то возвращаем текущий элемент, увеличенный на 1. Покажем, что наш алгоритм получает результат не хуже этого. В первом цикле алгоритм расставил все элементы, которые возможно, на ``правильные'' места. Во втором цикле алгоритм нашел позицию первого ``неправильно'' стоящего элемента и потом вычислил минимальное число (добавил 1 к позиции). Видно, что результат не хуже ответа.

      Асимптотика: первый цикл работает за $\O(n)$, потому что, если приходит число, которое превосходит размер массива, то ничего не происходит, и итератор просто увеличивается на 1. Или, если приходит число, которое можно поставить на ``правильное'' место, то происходит свап за $\O(1)$, потом снова проверяется, можно ли поставить очередное число на нужное место. В худшем случае этот внутренний цикл на какой-нибудь итерации проделает $\O(n)$ шагов, но в таком случае на всех остальных итерациях внешнего цикла ничего происходить не будет, поэтому можно сказать, что в сумме будет ($\O(n)+\O(n)=\O(n)$). Второй цикл работает за $\O(n)$ ($n \cdot \O(1)$). В конце возвращение значения за $\O(1)$. В сумме:
      \begin{equation}
        T(n) = \O(n) + \O(n) + \O(1) = \O(n).
      \end{equation}
    \end{solution}

  \item
    Даны $n$ точек на плоскости ($x_i, y_i$) с весами $w_i \ge 0$. Найти точку ($x^{*}, y^{*}$):
    \begin{enumerate}
      \item \onlygroup{Крыштаповича} $\max\limits_i \Bigl[ \max(|x_i-x^{*}|, |y_i-y^{*}|) \Bigr] \rightarrow \min$
      \begin{solution}
        Если рассмотреть одномерную задачу, когда даны $n$ точек $x_i$ и хочется найти $x^{*}$ такую, что
        \begin{equation}
          \max\limits_i(|x_i-x^{*}|) \rightarrow \min,
        \end{equation}
        то очевидно, что интересующая нас точка будет находиться ровно посередине между самой левой точкой и самой правой точкой, потому что, если начать сдигаться влево или вправо, то максимальный из модулей разностей будет увеличиваться, так как одно из расстояний (до крайнего левого или до крайнего правого) будет увеличиваться. Поэтому нужно брать именно середину. Середину можно получить следующим образом: берем массив точек $[x_i]$ и сортируем его, получаем массив точек $[x_i']$, в котором выполняется свойство
        \begin{equation}
          x_{i}' \leq x_{i+1}'.
        \end{equation}
        Теперь центральную точку вычислить легко через полусумму граничных точек:
        \begin{equation}
          x^{*} = \frac{x_{0}' + x_{n-1}'}{2}.
        \end{equation}
        Вернемся к исходной задаче: сначала по алгоритму выше найдем $x^{*}$, потом $y^{*}$:
        \begin{equation}
          \begin{cases}
            x^{*} = \dfrac{x_{0}' + x_{n-1}'}{2}, \\
            y^{*} = \dfrac{y_{0}' + y_{n-1}'}{2}.
          \end{cases}
        \end{equation}
      \end{solution}
      \item \onlygroup{Крыштаповича} $\max\limits_i \Bigl[ |x_i-x^{*}|+|y_i-y^{*}|  \Bigr] \rightarrow \min$
      \begin{solution}
        Кажется, что здесь подойдет то же самое решение, которое было предложено в предыдущей задаче, потому что сумма двух неотрицательных чисел будет стремиться к минимуму (нулю) тогда, когда оба слагаемых будут стремиться к нулю. Понятно, что центр (или медиана) будет гарантировать, что каждое из слагаемых $|x_i-x^{*}|$ и $|y_i-y^{*}|$ будет минимальным, поэтому и сумма будет минимальной.
      \end{solution}
      \item \onlygroup{Крыштаповича} $\sum\limits_i \Bigl[ (x_i-x^{*})^2+(y_i-y^{*})^2  \Bigr] \rightarrow \min$
      \begin{solution}
        Здесь наверное нужно посчитать частные производные по $x$ и $y$, приравнять к нулю и получить $(x^{*}, y^{*})$.
      \end{solution}
      \item $\sum\limits_i \Bigl[ w_i(|x_i-x^{*}|+|y_i-y^{*}|) \Bigr] \rightarrow \min$
      \item $\max\limits_i \Bigl[ w_i(|x_i-x^{*}|+|y_i-y^{*}|) \Bigr] \rightarrow \min$.
    \end{enumerate}

  \item
    В матрице $Q$ из натуральных чисел размера $N \times N$ найти подматрицу размера $H \times W$
    с максимальной медианой. $H, W$ --- нечётные.
    \begin{enumerate}
      \item $\O(N^2 \log{Q_{\max}})$. Здесь $Q_{\max}$ --- максимальный элемент матрицы.
      \item $\O(N^2 \log{N})$.
    \end{enumerate}

% \subsection*{Дополнительные задачи}
%   \item
%     Будем строить сортирующую сеть $M(n,m)$ для слияния двух отсортированных массивов
%     размера n и m следующим образом. Разделим элементы обоих массивов на четные и
%     нечетные, рекурсивно сольем отдельно четные, отдельно нечетные, в конце добавим
%     компараторы для элементов $2k$ и $2k + 1$ (см картинку):
    
%     \includegraphics[width=\textwidth]{source/sorting_network.png}
    
%     \begin{enumerate}
%      \item Покажите, что такая сеть действительно сливает два отсортированных массива.
%          \item Каково будет общее число компараторов в такой сети?
%      \item Какова будет глубина такой сети?
%     \end{enumerate}

\end{enumerate}