\section{Асимптотика}
\begin{enumerate}
  \item	
	Эквивалентны ли следующие факты?
	\begin{itemize}
		\item $f = \Theta(g)$
		\item $\exists C,  0 < C < \infty : \lim\limits_{n \to \infty} \frac{f(n)}{g(n)} = C$
	\end{itemize}

  \begin{proof}
    По определению из лекции:
    \begin{eqnarray}
      f(n) = \Theta(g(n)), \text{ если } f(n) = \O(g(n)) \text{ и } g(n) = \O(f(n)), \\
      f(n) = \O(g(n)), \text{ если } \exists C > 0, N \in \N: \forall n \geq N \quad f(n) \leq C \cdot g(n), \\
      g(n) = \O(f(n)), \text{ если } \exists C > 0, N \in \N: \forall n \geq N \quad g(n) \leq C \cdot f(n).
    \end{eqnarray}
    Таким образом, мы имеем систему:
    \begin{equation}
      \begin{cases}
        f(n) \leq C_1 \cdot g(n), \\
        g(n) \leq C_2 \cdot f(n),
      \end{cases}
    \end{equation}
    где первое уравнение выполняется с некоторого $n_1$, а второе с некоторого $n_2$. Так как $f$ и $g$ отображают в положительные числа, то можем поделить без смены знака неравенства:
    \begin{equation}
      \begin{cases}
        \dfrac{f(n)}{g(n)} \leq C_1, \\ \qquad\qquad\qquad\qquad\qquad\Rightarrow\\
        \dfrac{f(n)}{g(n)} \geq C_2,
      \end{cases}
    \end{equation}
    \begin{equation}
      C_2 \leq \dfrac{f(n)}{g(n)} \leq C_1,
    \end{equation}
  \end{proof}

  \item
    Дайте ответ для двух случаев $\mathbb{N} \to \mathbb{N}$ и $\mathbb{N} \to \mathbb{R}_{>0}$:
    \begin{enumerate}
      \item
        Если в определении $\O$ опустить условие про $N$ (т.е. оставить
        просто $\forall n$), будет ли полученное определение эквивалентно исходному?

      \begin{proof}
        Исходное определение:
        \begin{equation}
          f(n) = \O(g(n)), \quad \text{если } \quad \exists C > 0, N \in \N: \forall n \geq N \quad f(n) \leq C \cdot g(n)
        \end{equation}
        Рассмотрим случай, когда отображение $\N \to \N$. Если опустить условие про $N$, то определения останутся эквивалентными, потому что в конечном итоге мы будем иметь неравенство:
        \begin{equation}
          f(n) \leq C \cdot g(n),
        \end{equation}
        которое просто сводится к
        \begin{equation}
          C_1 \leq C \cdot C_2,
        \end{equation}
        где $C_1, C_2 \in \N$. Очевидно, что мы всегда можем подобрать такую $C$, чтобы неравенство выполнялось, например:
        \begin{equation}
          C = \dfrac{C_1}{C_2} + 1.
        \end{equation}
        Если отображение $\N \to \mathbb{R}_{>0}$, то:
        \begin{equation}
          C_1' \leq C \cdot C_2',
        \end{equation}
        где $C_1', C_2' \in \mathbb{R}_{>0}$. Если $C_1, C_2$ -- бесконечно малые (или бесконечно большие), $C$ можно легко подобрать. Если $C_1$ -- бесконечно большая, а $C_2$ -- бесконечно малая, то $C$ можно взять как бесконечно большую более высокого порядка по сравнению с $\frac{C_1}{C_2}$.

        Так как $C$ всегда можно найти, то определение без конкретного $N$ оказывается эквивалентным исходному.
      \end{proof}
      \item Тот же вопрос про $o$.
      
      В данном случае неравенства будут строгими, но, делая те же самые рассуждения, которые были сделаны в пункте (a), можно понять, что $C$ тоже всегда удастся подобрать.
      \begin{equation}
        f(n) = o(g(n)), \quad \text{если } \quad \exists C > 0, N \in \N: \forall n \geq N \quad f(n) < C \cdot g(n)
      \end{equation}
      Определения будут эквивалентными.
    \end{enumerate}

  \item
    Продолжим отношение ``$\preceq$'' на функциях до отношения на классах эквивалентности по отношению эквивалентности ``$\sim$'', введённому на практике. Правда ли, что получится отношение \textit{линейного порядка} (то есть
    $\forall f, g: (f \preceq g) \lor (g \preceq f)$)?
    \begin{proof}
      Нет, не правда. Эквивалентность $f \sim g$ означает, что функции $f$ и $g$ асимптотически имеют одинаковую скорость роста. При этом, такое поведение можно гарантировать, если одновременно выполнены два условия: $(f \preceq g)$ и $(g \preceq f)$. В математических терминах это можно записать через пересечение:
      \begin{equation}
        \forall f, g: (f \preceq g) \land (g \preceq f)
      \end{equation}
      Если бы мы оставили объединение, то мы бы получили положительный ответ (что $f \sim g$), даже если только одно из утверждений ($(f \preceq g)$ или $(g \preceq f)$) было бы верным, ну а это неверно.
    \end{proof}

  \item
    Докажите, или приведите контрпример:
    \begin{enumerate}
      \item $g(n) = o(f(n)) \Rightarrow f(n) + g(n) = \Theta(f(n))$
      \item $f(n) = \O(g(n)) \Leftrightarrow f(n) = o(g(n)) \lor f(n) = \Theta(g(n))$
    \end{enumerate}

  \item Заполните табличку и поясните (особенно строчки 4 и 7):
    $$
    \begin{array}{|cc|c|c|c|c|c|}
      \hline
      A & B & \O & o & \Theta & \omega & \Omega \\
      \hline
      n & n^2 & + & + & - & - & - \\
      \log^k n & n^{\epsilon} & & & & & \\
      n^k & c^n & & & & & \\
      \sqrt{n} & n^{\sin n} & & & & & \\
      2^n & 2^{n \slash 2} & & & & & \\
      n^{\log m} & m^{\log n} & & & & & \\
      \log (n!) & \log(n^n) & & & & & \\
      \hline
    \end{array}
    $$
    Здесь все буквы, кроме $n$, --- положительные константы.

\subsection*{Дополнительные задачи}
    
    \item  
    Считайте здесь, что функции здесь $\mathbb{N} \to \mathbb{N}$ и что $\forall n : f(n) > 1 \land g(n) > 1$.
    \begin{enumerate}
      \item $f(n) = \Omega(f(n \slash 2)$)?
      \item $f(n) = \O(g(n)) \Rightarrow \log f(n) = \O(\log g(n))$?
      \item $f(n) = \O(g(n)) \Rightarrow 2^{f(n)} = \O(2^{g(n)})$?
      \item $f(n) = o(g(n)) \Rightarrow \log f(n) = o(\log g(n))$?
      \item $f(n) = o(g(n)) \Rightarrow 2^{f(n)} = o(2^{g(n)})$?
    \end{enumerate}
    
	\item
	Упорядочите функции по скорости роста и обозначьте неравенства между соседями. 
	Укажите, в каких неравенствах $f = o(g)$, а в каких $f = \Theta(g)$
	$$
	\begin{array}{cccccc}
	\log(\log^* n) & 2^{log^* n} & (\sqrt{n})^{\log n} & n^2 & n! & (\log n)! \\
	(3 \slash 2)^n & n^3 & \log^2 n & \log n! & 2^{2^n} & n^{1 \slash \log n} \\
	\ln \ln n & \log^* n & n \cdot 2^n & n^{\log \log n} & \ln n & 1 \\
	2^{\ln n} & (\log n)^{\log n} & e^n & 4^{\log n} & (n + 1)! & \sqrt{\log n} \\
	\log^* \log n & 2^{\sqrt{2 \log n}} & n & 2^n & n \log n & 2^{2^{n + 1}}					
	\end{array}
	$$
	Примечание: $\log^*(n) = \left\{
	\begin{array}{ll}
	0 & \texttt{ если } n \leq 1;\\
	1 + \log^*(\log n) & \texttt{ иначе.}
	\end{array}
	\right.$
\end{enumerate}

\clearpage
