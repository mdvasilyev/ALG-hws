\section{Сортировки и порядковые статистики}
% \subsection{Практика}
% \begin{enumerate}
%   \item
%     Дана последовательность из $n$ чисел, нужно за один проход и $\O(n)$ времени найти в ней
%     $k$ минимумов, используя $\O(k)$ дополнительной памяти.

%   \item
%     Работает ли аналогичная оценка времени работы детерминированного алгоритма
%     поиска порядковой статистики, если вместо пятерок разбивать элементы на
%     \begin{enumerate}
%       \item семерки.
%       \item тройки.
%     \end{enumerate}

%   \item
%     Придумайте, как добиться от \texttt{QuickSort} времени $\O(n \log{n})$
%     в худшем случае.

%   \item
%     Дан массив из n чисел от 1 до k, разработайте структуру данных, которая за $\O(1)$
%     отвечает на online запросы вида «Сколько в массиве элементов от a до b?». Время на
%     предподсчет $\O(n + k)$.

%   \item
%     Как с помощью цифровой сортировки сортировать строки над константным алфавитом
%     в лексикографическом порядке за $\O(\sum |s_i|)$?

%   \item
%     Даны $n$ точек на плоскости ($x_i, y_i$). Найти точку ($x^{*}, y^{*}$):
%     \begin{enumerate}
%       \item $\max\limits_i \Bigl[ \max(|x_i-x^{*}|, |y_i-y^{*}|) \Bigr] \rightarrow \min$
%       \item $\max\limits_i \Bigl[ |x_i-x^{*}|+|y_i-y^{*}|  \Bigr] \rightarrow \min$
%       \item $\sum\limits_i \Bigl[ (x_i-x^{*})^2+(y_i-y^{*})^2  \Bigr] \rightarrow \min$
%       \item $\sum\limits_i \Bigl[ |x_i-x^{*}|+|y_i-y^{*}| \Bigr] \rightarrow \min$
%       \item $\sum\limits_i \Bigl[ \max(|x_i-x^{*}|, |y_i-y^{*}|) \Bigr] \rightarrow \min$
%     \end{enumerate}

%   \item
% 	Даны два массива из положительных чисел $a$ и $b$, $|a| = |b| = n$.
% 	Выбрать массив $p$ из $k$ различных чисел от $1$ до $n$ так, чтобы
% 	$\frac{\sum_{i=1}^{k}a_{p_i}}{\sum_{i=1}^{k}b_{p_i}} \to \max$. %$\O(n \log{n})$.

% \subsection*{Дополнительные задачи}

%   \item
%     Возьмем массив a из n элементов, каждый из которых — это число $1$ до $n$.
%     Циклический сдвиг номер $i$ -- это последовательность
%     $a_i, a_{i+1}, \dots, a_{n-1}, a_0, a_1, \dots, a_{i-1}$. 
%     \begin{enumerate}
%       \item Предложите алгоритм сортировки циклических сдвигов в лексикографическом
%       порядке за время $O(n^2)$.
%       \item Предложите алгоритм сортировки циклических сдвигов за $\O(n \log n)$.
%     \end{enumerate}

%   \item
%     Пусть мы хотим показать, что некоторая сортирующая сеть сливает два отсортированных
%     массива в один. Покажите, что для этого достаточно показать, что она сливает массивы,
%     состоящие из $0$ и $1$.

% \end{enumerate}

% \pagebreak
\subsection{Домашнее задание}
\begin{enumerate}
  \item
    Есть улица длины $l$, которая освещается $n > 0$ фонарями, $i$-й фонарь находится в
    точке с вещественной координатой $a_i$. Фонарь освещает все точки улицы, которые находятся от него на расстоянии не больше d, где $d$ — некоторое
    положительное число, общее для всех фонарей.
    Найдите минимальное $d$, при котором вся улица освещена. $\O(n)$, $a_i$ не отсортированы.

  \item
    Есть массив $a_i$, состоящий из $n$ неотрицательных целых чисел. Найти минимальное натуральное
    число, которого нет в массиве, за $\O(n)$ времени и $\O(1)$ дополнительной памяти.
    Входной массив доступен для записи. \\
    Замечание: $0 \leq a_i < 2^w$, где $w$ -- размер одной ячейки массива в битах.

  \item
    Даны $n$ точек на плоскости ($x_i, y_i$) с весами $w_i \ge 0$. Найти точку ($x^{*}, y^{*}$):
    \begin{enumerate}
      \item \onlygroup{Крыштаповича} $\max\limits_i \Bigl[ \max(|x_i-x^{*}|, |y_i-y^{*}|) \Bigr] \rightarrow \min$
      \item \onlygroup{Крыштаповича} $\max\limits_i \Bigl[ |x_i-x^{*}|+|y_i-y^{*}|  \Bigr] \rightarrow \min$
      \item \onlygroup{Крыштаповича} $\sum\limits_i \Bigl[ (x_i-x^{*})^2+(y_i-y^{*})^2  \Bigr] \rightarrow \min$
      \item $\sum\limits_i \Bigl[ w_i(|x_i-x^{*}|+|y_i-y^{*}|) \Bigr] \rightarrow \min$
      \item $\max\limits_i \Bigl[ w_i(|x_i-x^{*}|+|y_i-y^{*}|) \Bigr] \rightarrow \min$.
    \end{enumerate}

  \item
    В матрице $Q$ из натуральных чисел размера $N \times N$ найти подматрицу размера $H \times W$
    с максимальной медианой. $H, W$ --- нечётные.
    \begin{enumerate}
      \item $\O(N^2 \log{Q_{\max}})$. Здесь $Q_{\max}$ --- максимальный элемент матрицы.
      \item $\O(N^2 \log{N})$.
    \end{enumerate}

% \subsection*{Дополнительные задачи}
%   \item
%     Будем строить сортирующую сеть $M(n,m)$ для слияния двух отсортированных массивов
%     размера n и m следующим образом. Разделим элементы обоих массивов на четные и
%     нечетные, рекурсивно сольем отдельно четные, отдельно нечетные, в конце добавим
%     компараторы для элементов $2k$ и $2k + 1$ (см картинку):
    
%     \includegraphics[width=\textwidth]{source/sorting_network.png}
    
%     \begin{enumerate}
%      \item Покажите, что такая сеть действительно сливает два отсортированных массива.
%          \item Каково будет общее число компараторов в такой сети?
%      \item Какова будет глубина такой сети?
%     \end{enumerate}

\end{enumerate}